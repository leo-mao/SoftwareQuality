% Einstellen der Dokumentenklasse (möglich: book, report, article, dinbrief, beamer, scrbook, scrreprt, scrartcl)
% Einstellen der Dokumentoptionen (möglich: 11pt, twocolumn, oneside, titlepage, landscape, a4paper)
\documentclass[12pt,a4paper]{article}
    
% Sprach- und Kodierungseinstellungen
\usepackage[utf8]{inputenc}
\usepackage[ngerman]{babel}

% Literaturverzeichnis
\usepackage{cite}

% Stil des Literaturverzeichnisses festlegen (möglich: plain, abbrv, alpha, unsrt, natbib)
\bibliographystyle{plain}

% Grafikpakete für Bilder und Vektorgraphiken
\usepackage{graphicx}
\usepackage{float}

% Weitere Pakete
\usepackage{amsmath}
\usepackage{amsfonts}
\usepackage{amssymb}

% Für Kopfzeile
\usepackage{fancyhdr}

% Für Tabellen
\usepackage{booktabs}
\usepackage[table,xcdraw]{xcolor}

% Dokumentformatierung (möglich: plain, empty, headings, myheadings)
\pagestyle{plain}

% Seitenzahlenstil (möglich: arabic, roman, Roman, alph, Alph)
\pagenumbering{arabic}

% Hyperlink Package
\usepackage{hyperref}

% Für URLs
\usepackage{url}
\def\UrlBreaks{\do\/\do-}

% Farbe der Hyperlinks anpassen
\hypersetup{
    colorlinks,
    linkcolor={black},
    citecolor={blue},
    urlcolor={black}
}

% Footnote in table
\usepackage{tablefootnote}

% Fancy Pagestyle ändern
\fancypagestyle{plain}{}
\renewcommand{\headrulewidth}{0pt}
\renewcommand{\footrulewidth}{0.4pt}


\title{	
\textbf{Praktikum Aufgabe\\
		Äquivalenzklassen und Grenzwertanalyse\\}
	    Software Qualitätssicherung\\
		SS2018\\
		Hochschule Emden/Leer}
\author{Yang Mao}
\date{\today}

% Fußzeile
\fancyfoot{}		% Fußzeile leeren
\lfoot{Software Qualitätssicherung}
\rfoot{Praktikum Aufgabe}

% Color for listing
\usepackage{color}

\definecolor{pblue}{rgb}{0.13,0.13,1}
\definecolor{pgreen}{rgb}{0,0.5,0}
\definecolor{pred}{rgb}{0.9,0,0}
\definecolor{pgrey}{rgb}{0.46,0.45,0.48}

% Listing

\usepackage{listings}
\lstset{language=Java,
	showspaces=false,
	showtabs=false,
	numbers=left,
	stepnumber=5,    
	firstnumber=0,
	numberfirstline=true
	breaklines=true,
	showstringspaces=false,
	breakatwhitespace=true,
	commentstyle=\color{pgreen},
	keywordstyle=\color{pblue},
	stringstyle=\color{pred},
	basicstyle=\ttfamily,
	moredelim=[il][\textcolor{pgrey}]{\$\$},
	moredelim=[is][\textcolor{pgrey}]{\%\%}{\%\%}
}

%\makeindex
              
% Start der Inhaltsumgebung
\begin{document}
    % Titel anzeigen
    \maketitle
    \newpage
    
    \tableofcontents
    \newpage
	% Dokumentinformationen
	
\section{Übung 1}
\begin{table}[H]
	\centering
	\small
	\setlength
	\tabcolsep{2pt}
	\begin{tabular}{l|l|l|l|l|l}
		Lfd.Nr. & Eingabe & Gültige Äquivalenzklasse & Ungültige Äquivalenzklasse & Soll-Ergebnis& Repräsentant\\ \hline
		1 & Alter & $18 < Alter \leq 65$ & 	$Alter \leq 18 \bigcup Alter >$ 65 & Richtig & 19\\
		2 & Alter & $18 < Alter \leq 65$ & 	$Alter \leq 18 \bigcup Alter >$ 65 & Falsch & 18\\
		3 & Alter & $18 < Alter \leq 65$ & 	$Alter \leq 18 \bigcup Alter >$ 65 & Richtig & 65\\
		4 & Alter & $18 < Alter \leq 65$ & 	$Alter \leq 18 \bigcup Alter >$ 65 & Falsch &  64\\
		5 & Alter & $18 < Alter \leq 65$ & 	$Alter \leq 18 \bigcup Alter >$ 65 & Richtig &  30\\
	\end{tabular}
		\caption{Äquivalenzklassen}
\end{table}

\section{Übung 2}
\begin{table}[H]
	\centering
	\small
	\setlength
	\tabcolsep{2pt}
	\begin{tabular}{l|l|l|l|l|l}
		Lfd.Nr. & Eingabe & Gültige Äquivalenzklasse & Ungültige Äquivalenzklasse & Soll-Ergebnis& Repräsentant\\ \hline
		1 & Tag & $1 < Tag \leq 31$ & 	$Tag <1 \bigcup Tag >$ 31 & Richtig & 1\\
		2 & Tag & $1 < Tag \leq 31$ & 	$Tag <1 \bigcup Tag >$ 31 & Falsch & 0\\
		3 & Tag & $1 < Tag \leq 31$ & 	$Tag <1 \bigcup Tag >$ 31 & Richtig & 31\\
		4 & Tag & $1 < Tag \leq 31$ & 	$Tag <1 \bigcup Tag >$ 31 & Falsch & 32\\
		5 & Tag & $1 < Tag \leq 31$ & 	$Tag <1 \bigcup Tag >$ 31 & Richtig & 30\\		
\end{tabular}
	\caption{Äquivalenzklassen}
\end{table}

\section{Übung 3}

\begin{table}[H]
	\centering
	\small
	\setlength
	\tabcolsep{2pt}
	\begin{tabular}{l|l|l|l|l|l}
		Lfd.Nr. & Eingabe & Gültige Äquivalenzklasse & Ungültige Äquivalenzklasse & Soll-Ergebnis& Repräsentant\\ \hline
		1 & Zahlungsweise & $Eingabe \subset a,b,c,d$ & $Eingabe \not\subset a,b,c,d$	  & Richtig & a\\
		1 & Zahlungsweise & $Eingabe \subset a,b,c,d$ & $Eingabe \not\subset a,b,c,d$	  & Richtig & b\\
		1 & Zahlungsweise & $Eingabe \subset a,b,c,d$ & $Eingabe \not\subset a,b,c,d$	  & Richtig & c\\
		1 & Zahlungsweise & $Eingabe \subset a,b,c,d$ & $Eingabe \not\subset a,b,c,d$	  & Richtig & d\\
		1 & Zahlungsweise & $Eingabe \subset a,b,c,d$ & $Eingabe \not\subset a,b,c,d$	  & Falsch & e\\
	\end{tabular}
	\caption{Äquivalenzklassen}
	\begin{text}
		\small
		$a$= Jahrlich, $b$= Halbjährlich, $c$= Vierteljährlich, $d$= Monatlich, $e$= Taglich
	\end{text}
\end{table}


\section{Übung 4}
\begin{table}[H]
	\centering
	\small
	\setlength
	\tabcolsep{2pt}
	\begin{tabular}{l|l|l|l|l|l}
		Lfd.Nr. & Eingabe & Gültige Äquivalenzklasse & Ungültige Äquivalenzklasse & Soll-Ergebnis& Repräsentant\\ \hline
		1 & $b$  & $Eingabe \subset a$ & $Eingabe \not\subset a$& Richtig & ``A1'' \\
		2 & Zahl & $Eingabe \subset a$ & $Eingabe \not\subset a$ & Falsch & 100\\
	\end{tabular}
	\caption{Äquivalenzklassen}
	\begin{text}
		\small
	$a$= Eingabe mit einen angefangen Buchstaben, $b$= Ein String mit einem Buchstaben Buchstaben
	\end{text}
\end{table}

\section{Übung 5}
\begin{table}[H]
	\centering
	\small
	\setlength
	\tabcolsep{2pt}
	\begin{tabular}{l|l|l|l|l|l}
		Lfd.Nr. & Eingabe & Gültige Äquivalenzklasse & Ungültige Äquivalenzklasse & Soll-Ergebnis& Repräsentant\\ \hline
		1 & Wert & $1 \leq Wert \leq 99$ & 	$Tag < 1 \bigcup Tag >$ 99 & Richtig & 1\\
		2 & Wert & $1 \leq Wert \leq 99$ & 	$Tag < 1 \bigcup Tag >$ 99 & Falsch &0\\
		3 & Wert & $1 \leq Wert \leq 99$ & 	$Tag < 1 \bigcup Tag >$ 99 & Richtig & 99\\
		4 & Wert & $1 \leq Wert \leq 99$ & 	$Tag < 1 \bigcup Tag >$ 99 & Falsch & 100\\
	\end{tabular}
	\caption{Äquivalenzklassen}
\end{table}

\section{Übung 6}
\begin{table}[H]
	\centering
	\small
	\setlength
	\tabcolsep{2pt}
	\begin{tabular}{l|l|l|l|l|l}
		Lfd.Nr. & Eingabe & Gültige Äquivalenzklasse & Ungültige Äquivalenzklasse & Soll-Ergebnis& Repräsentant\\ \hline
		1 & Holzart & 1.Eiche 2.Buch 3.Kiefer  & 4.Andere & Richtig & Holzart\\ 
		2 & Länge & 5. $100 \leq$ Länge $\leq$ $500$ & 6. Länge $< 100 \bigcup$  Länge $> 500$ & Richtig& 100\\
		3 & Anzahl & 7. $1 \leq$ Anzahl $\leq$ $9999$ &8.  Anzahl $< 1 \bigcup$  Länge $> 9999$ & Richtig& 100\\
		4 & Auftragsnummer & 9. Erstes Zeichen ist H & 10. Erstes Zeichen ist nicht H& Richtig & H777
	\end{tabular}
	\caption{Äquivalenzklassen}
\end{table}

\section{Übung 7}
\begin{table}[H]
	\centering
	\small
	\setlength
	\tabcolsep{2pt}
	\begin{tabular}{l|l|l|l|l|l}
		Lfd.Nr. & Eingabe & Gültige Äquivalenzklasse & Ungültige Äquivalenzklasse & Soll-Ergebnis& Repräsentant\\ \hline
		1 & annualPercentageRate(apr) & [0..10] & andere & Richtig & 2\\
		2 & loanValue & [0..500.000,00] & andere & Richtig & 100\\
		3 & loanPeriod & [1..120] & andere & Richtig & 10\\
	\end{tabular}
	\caption{Äquivalenzklassen}
\end{table}
	1*1*1 = 1 gültiger Testfälle
	2*2*2 - 1 = 7 ungültiger Testfälle

\end{document}