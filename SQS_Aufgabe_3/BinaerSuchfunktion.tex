% Einstellen der Dokumentenklasse (möglich: book, report, article, dinbrief, beamer, scrbook, scrreprt, scrartcl)
% Einstellen der Dokumentoptionen (möglich: 11pt, twocolumn, oneside, titlepage, landscape, a4paper)
\documentclass[12pt,a4paper]{article}
    
% Sprach- und Kodierungseinstellungen
\usepackage[utf8]{inputenc}
\usepackage[ngerman]{babel}

% Literaturverzeichnis
\usepackage{cite}

% Stil des Literaturverzeichnisses festlegen (möglich: plain, abbrv, alpha, unsrt, natbib)
\bibliographystyle{plain}

% Grafikpakete für Bilder und Vektorgraphiken
\usepackage{graphicx}
\usepackage{float}

% Weitere Pakete
\usepackage{amsmath}
\usepackage{amsfonts}
\usepackage{amssymb}

% Für Kopfzeile
\usepackage{fancyhdr}

% Für Tabellen
\usepackage{booktabs}
\usepackage[table,xcdraw]{xcolor}

% Dokumentformatierung (möglich: plain, empty, headings, myheadings)
\pagestyle{plain}

% Seitenzahlenstil (möglich: arabic, roman, Roman, alph, Alph)
\pagenumbering{arabic}

% Hyperlink Package
\usepackage{hyperref}

% Für URLs
\usepackage{url}
\def\UrlBreaks{\do\/\do-}

% Farbe der Hyperlinks anpassen
\hypersetup{
    colorlinks,
    linkcolor={black},
    citecolor={blue},
    urlcolor={black}
}

% Fancy Pagestyle ändern
\fancypagestyle{plain}{}
\renewcommand{\headrulewidth}{0pt}
\renewcommand{\footrulewidth}{0.4pt}


\title{	
\textbf{Praktikum Aufgabe\\
		Binäre Suchfunktion\\}
	    Software Qualitätssicherung\\
		SS2018\\
		Hochschule Emden/Leer}
\author{Yang Mao}
\date{\today}

% Fußzeile
\fancyfoot{}		% Fußzeile leeren
\lfoot{Software Qualitätssicherung}
\rfoot{Praktikum Aufgabe}

% Color for listing
\usepackage{color}

\definecolor{pblue}{rgb}{0.13,0.13,1}
\definecolor{pgreen}{rgb}{0,0.5,0}
\definecolor{pred}{rgb}{0.9,0,0}
\definecolor{pgrey}{rgb}{0.46,0.45,0.48}

% Listing

\usepackage{listings}
\lstset{language=Java,
	showspaces=false,
	showtabs=false,
	numbers=left,
	stepnumber=5,    
	firstnumber=0,
	numberfirstline=true
	breaklines=true,
	showstringspaces=false,
	breakatwhitespace=true,
	commentstyle=\color{pgreen},
	keywordstyle=\color{pblue},
	stringstyle=\color{pred},
	basicstyle=\ttfamily,
	moredelim=[il][\textcolor{pgrey}]{\$\$},
	moredelim=[is][\textcolor{pgrey}]{\%\%}{\%\%}
}

%\makeindex
              
% Start der Inhaltsumgebung
\begin{document}
    % Titel anzeigen
    \maketitle
    \newpage
    
    \tableofcontents
    \newpage
	% Dokumentinformationen
	
\section{Testorakeln}
	\begin{enumerate}
		\item Objektfeld, ein geordnetes Feld (Array) von Objekten
		\item Das Ergebnisobjekt hat 2 Attribute,
		\\ $\rightarrow$ Index: die Stelle, wo das gesuchte in dem Feld gefunden wurde, -1 wenn das Gesuchte nicht	gefunden wurde.
		\\ $\rightarrow$ Gefunden:  ein boolescher Wert true/false, je nach dem, ob das Gesuchte gefunden wurd.
	\end{enumerate}
\section{Äquivalenzklassen}

\begin{table}[H]
	\centering
	\small
	\setlength
	\tabcolsep{2pt}
	\begin{tabular}{l|l|l}
		Eingabe & gültige Äquivalenzklassen & ungültige Äquivalenzklassen \\ \hline
		Objektfeld & Array mit geordnetes Objekte & Array mit  unsortierte Objekte \\ [-3pt]
		Schlüssel & ein Objekte & andere Eingaben \\ 
		Leeres Ergebnis & ein Ergbnisobjekt mit 2 Felder, Index und Gefunden & andere Eingaben	
	\end{tabular}
		\caption{Äquivalenzklassen}
\end{table}
    	
\section{Testdaten}
   \begin{table}[H]
   	\centering
   	\begin{tabular}{l|l}
   		Testfälle & 1 \\ \hline
   		Objektfeld & [2, 3, 1] \\ \hline
   		Schlüssel & 2 \\ \hline
   		Leeres Ergebnis & {Index:null, Gefunden: null} \\ \hline
   		Test Ergebnis & Abbruch 
   	\end{tabular}
   		\caption{Testdaten}
   \end{table}
    
\section{Anhang}
\begin{lstlisting}
import java.util.Random;
import java.util.ArrayList;
import java.util.Collections;

public class BinaerSuchfunktion {
public static void BinaerSuchfunktion(int key, int[] field, int low, int high, Result result){
if (low == high) {
if (field[low] == key) {
result.found = true;
result.index = low;
}
return;
}
int mid = (low + high) / 2;
if (field[mid] == key){
result.found = true;
result.index = mid;
}
else if (field[mid] > key) {
BinaerSuchfunktion(key, field, low, mid, result);
}
else if (field[mid] < key) {
BinaerSuchfunktion(key, field, mid + 1, high, result);
}
}
public static void main(String[] args){
int[] field;
Random random = new Random();
int num = random.nextInt(20);
ArrayList arrayList = new ArrayList<Integer>();
Result result = new Result();


for (int i = 0; i < num; i++ ){
arrayList.add(random.nextInt(1000));
}

Collections.sort(arrayList);

System.out.println(arrayList);
field = arrayList.stream().mapToInt(i -> (int) i).toArray();

int index = random.nextInt(num);
System.out.printf("Richtige Index: %d, Gesuchte Zahl: %d", index + 1, field[index]);
System.out.println();
BinaerSuchfunktion(field[index], field, 0, num - 1, result);
System.out.println(result);
}
}
\end{lstlisting}
\end{document}